\documentclass[paper=a4, fontsize=10pt]{article}
\usepackage{array, xcolor, lipsum, bibentry}
\usepackage[margin=3cm]{geometry}
\usepackage[utf8]{inputenc}
\usepackage[T1]{fontenc}
\usepackage[english,norsk]{babel}
\usepackage[colorlinks=true, a4paper=true, pdfstartview=FitV,
            linkcolor=blue, citecolor=blue, urlcolor=blue]{hyperref}

\title{\bfseries\Huge Nils Peder Korsveien}
\author{Curriculum Vitae}
\date{}

\definecolor{lightgray}{gray}{0.8}
\newcolumntype{L}{>{\raggedleft}p{0.14\textwidth}}
\newcolumntype{R}{p{0.8\textwidth}}
\newcommand\VRule{\color{lightgray}\vrule width 0.5pt}
\begin{document}
\maketitle
\vspace{0.3em}
\section*{Om meg}
Jeg er 28 år og tar master i informatikk samtidig som jeg jobber deltid
hos PayEx som systemutvikler. Jeg er nysgjerrig på nye teknologier og
brenner for informatikk og design.

\section*{Personlige opplysninger}
\begin{tabular}{L!{\VRule}R}
Født&23.07.1985\\
Sivilstatus&Ugift\\
Adresse&Moldegata 15, H0702, 0445 Oslo\\[5pt]
E-post&\href{mailto:nilspk@ifi.uio.no}{nilspk@ifi.uio.no}\\
Mobil&470 57 184\\[5pt]
LinkedIn&\href{http://lnkd.in/JD8kz5}{http://lnkd.in/JD8kz5}\\
Github&\href{http://github.com/nilspk}{http://github.com/nilspk}\\
\end{tabular}

\section*{Jobberfaring}
\begin{tabular}{L!{\VRule}R}
2013--&{\bf Systemutvikler hos PayEx}\\
&
Dette er min nåværende arbeidsgiver etter at Mynt Betalingsterminaler
ble kjøpt opp av PayEx. Oppkjøpet førte med seg noen forandringer som at
vi nå bruker Scrum som utviklingsmetode og C++ som utviklingsspråk.
Hovedansvaret mitt er fortsatt grafikk og brukergrensesnitt. Jeg trives
veldig godt med å ha en rolle som både interaksjonsdesigner og
systemutvikler, siden det både gir variasjon og en helt unik
kompetanse.\\[5pt]

2012-2013&{\bf Juniorutvikler hos Mynt Betalingsterminaler}\\
&
Etter å ha jobbet her sommeren 2012 hvor vi utviklet et nytt
brukergrensesnitt som skal brukes på kortterminaler, ble dette min
faste deltidsjobb. Dette gir meg relevant og praktisk erfaring jeg veldig god
nytte av ved siden av studiene.
\\[5pt]

2009-2012&{\bf Kontormedarbeider hos Kredinor}\\
&
Deltidsansatt på kveldsteam hos Kredinor. Dette er en jobb jeg har hatt
ved siden av bachelorstudiene i Informatikk.Å balansere krevende skolearbeid med
deltidsjobb har vært en nyttig og lærerik erfaring.
\\[5pt]

2005-2008&{\bf Lagermedarbeider hos ASKO Hedmark}\\
&
Jobbet som plukker på grossistlager. Dette var en jobb jeg hadde fast
som sommerjobb over noen år, samt en periode på høsten 2008 hvor jeg var
fast ansatt.
\end{tabular}

\section*{Utdanning}
\begin{tabular}{L!{\VRule}R}
2009--&{\bf Master i Informatikk, Distribuerte Systemer, UiO}\\
2009--2012&{Bachelor i informatikk, Programmering og Nettverk, UiO}\\
2007--2008&{Ettårig, Nyere Historie, NTNU}\\
2006--2007&{Forkurs for ingeniørfag, HIST}\\
2001--2004&{Videregående, Medier og Kommunikasjon+Allmennfaglig påbygning}\\
\end{tabular}

\section*{Sideprosjekter}
\begin{tabular}{L!{\VRule}R}
2013&{\bf Boardwar}\\
&
Boardwar er en app til Android hvor du kan spille Reversi mot andre
brukere.
Dette var hobbyprosjektet til en venn og kollega. Han ansatte meg
til å designe brukergrensnittet og annen grafikk vi brukte til
nettsider og promo. Det var veldig moro å drive med dette ved siden av
arbdeid og skole, siden jeg også er veldig interessert i design og
brukeropplevelse.\\[5pt]

2012-2013&{\bf B.L.I.M.P - En selvflyvende robot!}\\
&
Dette var et prosjekt vi i utgangspunktet startet med i et skolefag om bruksorientert
design, men som var så spennende at vi tok det videre og presenterte det
på Idéfestivalen på Blinderen i September 2011. Video av dette kan sees \href{http://www.bit.ly/ifiblimp}{\tt{her}}. Prosjektet
har vært veldig lærerikt, og vi tok også med roboten på The Gathering i
2012 for å promotere institutt for informatikk.\\
\end{tabular}

\section*{Språk}
\begin{tabular}{L!{\VRule}R}
Norsk&Morsmål\\
Engelsk&Flytende\\
\end{tabular}

\section*{Kompetanse}
\begin{tabular}{L!{\VRule}R}
Språk&Clojure, Ruby, C, C++, Java, Python, Scheme, SQL, Bash, \LaTeX \\[5pt]
OS&Mac OS X, Linux, Windows\\[5pt]
VCS&Git, Mercurial, SVN\\[5pt]
Editor/IDE&Vim, IntelliJ, Eclipse\\[5pt]
Annet&Scrum, Adobe Photoshop og Illustrator, Arduino, zsh, tmux, gnuplot\\[5pt]
\end{tabular}

\section*{Referanse}
Oppgis etter behov.

\end{document}
